\documentclass[12pt,a4paper]{report}
\usepackage[utf8]{inputenc}
\usepackage{amsmath}
\usepackage{amsfonts}
\usepackage{amssymb}
\usepackage{polyglossia}

\usepackage[hyphens]{url}
\usepackage{titlesec}
\usepackage[affil-it]{authblk}
\usepackage{titling}

\setmainlanguage{hindi}
\newfontfamily\devanagarifont[Script=Devanagari]{Sanskrit 2003}
\titleformat*{\section}{\Large\devanagarifont}
\titleformat*{\subsection}{\large\devanagarifont}

\renewcommand{\maketitlehooka}{\devanagarifont}
\renewcommand\Authfont{\fontsize{12}{14.4}\devanagarifont}
\renewcommand\Affilfont{\fontsize{9}{10.8}\itshape}

\newcommand{\devanagarinumeral}[1]{%
  \devanagaridigits{\number\csname c@#1\endcsname}}

% renew all representation of counters
\renewcommand{\thesection}{\devanagarinumeral{section}}
\renewcommand{\thechapter}{\devanagarinumeral{chapter}}
\renewcommand{\thepage}{\devanagarinumeral{page}}
\renewcommand{\theenumi}{\devanagarinumeral{enumi}}


\author{अरिन्दम साहा}
\title{तर्कभाषाध्ययन}
\begin{document}

\maketitle

\chapter{उपोद्धातः}

\section{अनुबन्धचतुष्ट्यम्}

भारतीयपरम्परानुसारेण शास्त्रामुखे अनुबन्धचटुष्टयम् स्थापनीयम् । अनुबन्धचटुष्टये शास्त्रस्य प्रस्तावनारूपेण (१) विषयः, (२) प्रयोजनम्, (३) अधिकारी, (४) सम्बन्धः च इति चत्वारि अंशानि स्थाप्यन्ते । एतेषां प्रतिपत्त्यार्थं शास्त्रस्य आदिमं श्लोकं कर्त्रा श्रीकेशवमिश्रेण रचितम् ---

\begin{large}
\begin{center}
बालोऽपि यो न्यायनये प्रवेशमल्पेन वाञ्छत्यलसः श्रुतेन।\\
सङ्क्षिप्य युक्त्यन्विततर्कभाषा प्रकाश्यते तस्य कृते मयैषा।।
\end{center}

\begin{quote}
यः बालः अपि अलसः (अलसत्वात्) अल्पेन एव श्रुतेन न्यायनये प्रवेशं वाञ्छति (इच्छति), तस्य कृते सङ्क्षिप्य युक्त्यन्विततर्कभाषा मया प्रकाश्यते।
\end{quote}
\end{large}

तर्कभाषा इति शास्त्रस्य नाम। सा अध्ययनस्य विषयः। न्यायनयम् इति शास्त्रस्य प्रयोजनम्। बालः शास्त्रस्य अधिकारी। अपि च प्रकाशनं शास्त्रस्य शास्त्रकर्तुः मध्ये सम्बन्धः।

शास्त्रं सङ्क्षेपे वर्तते इति उक्तत्त्वात् मूलविषयस्य क्षयः नैव भवति इति द्योतुं युक्त्यन्वितेति पदं प्रयुक्तम्। सङ्क्षेपे कृतेऽपि तर्कभाषा युक्तियुक्ता। गौरवाभावे तत्त्वाभावः नैव कृतः।

\subsection*{पदचर्चा}

\begin{enumerate}
 \item बालः । नवछात्रः न तु शिशुः ।
  \item नीतिनये । नीयते विवक्षितोऽर्थः येन स न्यायः । नय इत्युक्ते नीतिः । न्यायस्य नयः न्यायनयः । तस्मिन् न्यायनये । न्यायनीत्याम् इति आशयः ।
\end{enumerate}

\section{षोडशपदार्थानि}

न्यायशास्त्रस्य प्रथमे सूत्रे षोडशपदार्थानि दत्तानि । तं सूत्रं स्वीकृत्य तदर्थं स्पष्टीकरोति शास्त्रकारः ---

\begin{large}
\begin{center}
प्रमाण-प्रमेय-संशय-प्रयोजन-दृष्टान्त-सिद्धान्तावयव-तर्क-निर्णय-वाद-जल्प-वितण्डा-हेत्वाभास-छल-जाति-निग्रहस्थानानां तत्त्वज्ञानात् निःश्रेयसाधिगमः इति न्यायस्यादिमं सूत्रम् । अस्यार्थः । प्रमानादिषोडशपदार्थानान्मोक्षप्राप्तिर्भवतीति ।
\end{center}
\end{large}

आदिमे न्यायसूत्रे षोडशपदार्थानाम् आवलिः दत्तः

\begin{enumerate}
 \item प्रमाण - प्रमाकरणं प्रमाणम्
 \item प्रमेय
 \item संशय
 \item प्रयोजन
 \item दृष्टान्त
 \item सिद्धान्त
 \item अववय
 \item तर्क
 \item निर्णय
 \item वाद
 \item जल्प
 \item वितण्डा
 \item हेत्वाभास
 \item छल
 \item जाति
 \item निग्रहस्थान
\end{enumerate}

अस्मिन् शास्त्रे एतेषां षोडशानां‌ पदार्थानां‌ विषये चर्चा कृता। तत्त्वज्ञानम् इत्युक्ते सम्यक् ज्ञानम्। पदार्थानां सम्यक् ज्ञाने को लक्ष्य इति चेदुच्यते निःश्रेयसाधिगमः। निःश्रेयसाधिगम इत्युक्ते मोक्षप्राप्तिः। सूत्रकारस्य मतेन षोडशपदार्थयुक्तस्य अस्य नीतिशास्त्रस्य प्रामुख्यं एतादृशं यत् तस्य सम्यक् ज्ञाने मोक्षप्राप्तिरपि भवति। 

\end{document}
