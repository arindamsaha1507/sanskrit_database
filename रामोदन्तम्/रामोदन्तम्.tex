\documentclass[a4paper,10pt]{report}
\usepackage[utf8]{inputenc}
\usepackage{amsmath}
\usepackage{amsfonts}
\usepackage{amssymb}
\usepackage{polyglossia}

\usepackage[hyphens]{url}
\usepackage{titlesec}
\usepackage[affil-it]{authblk}
\usepackage{titling}

\setmainlanguage{hindi}
\newfontfamily\devanagarifont[Script=Devanagari]{Sanskrit 2003}
% \newfontfamily\devanagarifont[Script=Devanagari]{Lohit Devanagari}
\titleformat*{\section}{\Large\devanagarifont}
\titleformat*{\subsection}{\large\devanagarifont}

\renewcommand{\maketitlehooka}{\devanagarifont}
\renewcommand\Authfont{\fontsize{12}{14.4}\devanagarifont}
\renewcommand\Affilfont{\fontsize{9}{10.8}\itshape}

\newcommand{\devanagarinumeral}[1]{%
  \devanagaridigits{\number\csname c@#1\endcsname}}

% renew all representation of counters
\renewcommand{\thesection}{\devanagarinumeral{section}}
\renewcommand{\thechapter}{\devanagarinumeral{chapter}}
\renewcommand{\thepage}{\devanagarinumeral{page}}
\renewcommand{\theenumi}{\devanagarinumeral{enumi}}

\newenvironment{moola}
{
~\\
\hrule
\begin{center}
\begin{LARGE}
}
{
\end{LARGE}
\end{center}
\hrule
~\\
}


\author{अरिन्दम साहा}
\title{रामोदन्तम्}
\begin{document}

\maketitle

\chapter{बालकाण्डम्}

\begin{moola}
श्रीपतिं प्रणिपत्याहं श्रीवत्साङ्कितवक्षसम् ।\\
श्रीरामोदन्तमाख्यास्ये श्रीवाल्मीकिप्रकीर्तितम् ॥ १॥ 
\end{moola}


\begin{moola}
पुरा विश्रवसः पुत्रो रावणो नाम राक्षसः ।\\
आसीदस्यानुजौ चास्तां कुम्भकर्णविभीषणौ ॥ २॥
\end{moola}

\begin{moola}
ते तु तीव्रेण तपसा प्रत्यक्षीकृत्य वेधसम् ।\\
वव्रिरे च वरानिष्टानस्मादाश्रितवत्सलात् ॥ ३॥
\end{moola}

\begin{moola}
रावणो मानुषादन्यैरवध्यत्वं तथानुजः ।\\
निर्देवत्वेच्छया निद्रां कुम्भकर्णोऽवृणीत च ॥ ४॥
\end{moola}

\begin{moola}
विभीषणो विष्णुभक्तिं वव्रे सत्त्वगुणान्वितः ।\\
तेभ्य एतान्वरान्दत्त्वा तत्रैवान्तर्दधे प्रभुः ॥ ५॥
\end{moola}

\begin{moola}
रावणस्तु ततो गत्वा रणे जित्वा धनाधिपम् ।\\
लङ्कापुरीं पुष्पकं च हृत्वा तत्रावसत्सुखम् ॥ ६॥
\end{moola}

\begin{moola}
यातुधानास्ततः सर्वे रसातलनिवासिनः ।\\
दशाननं समाश्रित्य लङ्कायां सुखमावसन् ॥ ७॥
\end{moola}

\begin{moola}
मन्दोदरीं मयसुतां परिणीय दशाननः ।\\
तस्यामुत्पादयामास मेघनादाह्वयं सुतम् ॥ ८॥
\end{moola}

\begin{moola}
रसां रसातलं चैव विजित्य स तु रावणः ।\\
लोकानाक्रमयन् सर्वाञ्जहार च विलासिनीः ॥ ९॥
\end{moola}

\begin{moola}
दूषयन्वैदिकं कर्म द्विजानर्दयति स्म सः ।\\
आत्मजेनान्वितो युद्धे वासवं चाप्यपीडयत् ॥ १०॥
\end{moola}

\begin{moola}
तदीयतरुरत्नानि पुनरानाय्य किङ्करैः ।\\
स्थापयित्वा तु लङ्कायामवसच्च चिराय सः ॥ ११॥
\end{moola}

\begin{moola}
ततस्तस्मिन्नवसरे विधातारं दिवौकसः ।\\
उपगम्योचिरे सर्वं रावणस्य विचेष्टितम् ॥ १२॥
\end{moola}

\begin{moola}
तदाकर्ण्य सुरैः साकं प्राप्य दुग्धोदधेस्तटम् ।\\
तुष्टाव च हृषीकेशं विधाता विविधैः स्तवैः ॥ १३॥
\end{moola}

\begin{moola}
आविर्भूयाथ दैत्यारिः पप्रच्छ च पितामहम् ।\\
किमर्थमागतोऽसि त्वं साकं देवगणैरिति ॥ १४॥
\end{moola}

\begin{moola}
ततो दशाननात्पीडामजस्तस्मै न्यवेदयत् ।\\
तच्छ्रुत्वोवाच धातारं हर्षयन्विष्टरश्रवाः ॥ १५॥
\end{moola}

\begin{moola}
अलं भयेनात्मयोने गच्छ देवगणैः सह ।\\
अहं दाशरथिर्भूत्वा हनिष्यामि दशाननम् ॥ १६॥
\end{moola}

\begin{moola}
आत्मांशैश्च सुराः सर्वे भूमौ वानररूपिणः ।\\
जायेरन्मम साहाय्यं कर्तुं रावणनिग्रहे ॥ १७॥
\end{moola}

\begin{moola}
एवमुक्त्वा विधातारं तत्रैवान्तर्दधे प्रभुः ।\\
पद्मयोनिस्तु गीर्वाणैः समं प्रायात्प्रहृष्टधीः ॥ १८॥
\end{moola}

\begin{moola}
अजीजनत्ततः शक्रो वालिनं नाम वानरम् ।\\
सुग्रीवमपि मार्ताण्डो हनुमन्तं च मारुतः ॥ १९॥
\end{moola}

\begin{moola}
पुरैव जनयामास जाम्बवन्तं च पद्मजः ।\\
एवमन्ये च विबुधाः कपीनजनयन्बहून् ॥ २०॥
\end{moola}

\begin{moola}
ततो वानरसङ्घानां वाली परिवृढोऽभवत् ।\\
अमीभिरखिलैः साकं किष्किन्धामध्युवास च ॥ २१॥
\end{moola}

\begin{moola}
आसीद्दशरथो नाम सूर्यवंशेऽथ पार्थिवः ।\\
भार्यास्तिस्रोऽपि लब्ध्वासौ तासु लेभे न सन्ततिम् ॥ २२॥
\end{moola}

\begin{moola}
ततः सुमन्त्रवचनादृष्यश‍ृङ्गं स भूपतिः ।\\
आनीय पुत्रकामेष्टिमारेभे सपुरोहितः ॥ २३॥
\end{moola}

\begin{moola}
अथाग्नेरुत्थितः कश्चिद्गृहीत्वा पायसं चरुम् ।\\
एतत्प्राशय पत्नीस्त्वमित्युक्त्वाऽदान्नृपाय सः ॥ २४॥
\end{moola}

\begin{moola}
तद्‍गृहीत्वा तदैवासौ पत्नीः प्राशयदुत्सुकः ।\\
ताश्च तत्प्राशनादेव नृपाद्गर्भमधारयन् ॥ २५॥
\end{moola}

\begin{moola}
पूर्णे कालेऽथ कौसल्या सज्जनाम्भोजभास्करम् ।\\
अजीजनद्रामचन्द्रं कैकेयी भरतं तथा ॥ २६॥
\end{moola}

\begin{moola}
ततो लक्ष्मणशत्रुघ्नौ सुमित्राजीजनत्सुतौ ।\\
अकारयत्पिता तेषां जातकर्मादिकं द्विजैः ॥ २७॥
\end{moola}

\begin{moola}
ततो ववृधिरेऽन्योन्यं स्निग्धाश्चत्वार एव ते ।\\
सकलासु च विद्यासु नैपुण्यमभिलेभिरे ॥ २८॥
\end{moola}

\begin{moola}
ततः कदाचिदागत्य विश्वामित्रो महामुनिः ।\\
ययाचे यज्ञरक्षार्थं रामं शक्तिधरोपमम् ॥ २९॥
\end{moola}

\begin{moola}
वसिष्ठवचनाद्रामं लक्ष्मणेन समन्वितम् ।\\
कृच्छ्रेण नृपतिस्तस्य कौशिकस्य करे ददौ ॥ ३०॥
\end{moola}

\begin{moola}
तौ गृहीत्वा ततो गच्छन्बलामतिबलां तथा ।\\
अस्त्राणि च समग्राणि ताभ्यामुपदिदेश सः ॥ ३१॥
\end{moola}

\begin{moola}
गच्छन्सहानुजो रामः कौशिकेन प्रचोदितः ।\\
ताटकामवधीद्धीमान् लोकपीडनतत्पराम् ॥ ३२॥
\end{moola}

\begin{moola}
ततः सिद्धाश्रमं प्राप्य कौशिकः सहराघवः ।\\
अध्वरं च समारेभे राक्षसाश्च समागमन् ॥ ३३॥
\end{moola}

\begin{moola}
राघवस्तु ततोऽस्त्रेण क्षिप्त्वा मारीचमर्णवे ।\\
सुबाहुप्रमुखान् हत्वा यज्ञं चापालयन्मुनेः ॥ ३४॥
\end{moola}

\begin{moola}
कौशिकेन ततो रामो नीयमानः सहानुजः ।\\
अहल्याशापनिर्मोक्षं कृत्वा सम्प्राप मैथिलम् ॥ ३५॥
\end{moola}

\begin{moola}
जनकेनार्चितो रामः कौशिकेन प्रचोदितः ।\\
सीतानिमित्तमानीतं बभञ्ज धनुरैश्वरम् ॥ ३६॥
\end{moola}

\begin{moola}
ततो दशरथं दूतैरानाय्य मिथिलाधिपः ।\\
रामादिभ्यस्तत्सुतेभ्यः सीताद्याः कन्यका ददौ ॥ ३७॥
\end{moola}

\begin{moola}
ततो गुरुनियोगेन कृतोद्वाहः सहानुजः ।\\
राघवो निर्ययौ तेन जनकेनोरु मानितः ॥ ३८॥
\end{moola}

\begin{moola}
तदाकर्ण्य धनुर्भङ्गमायान्तं रोषभीषणम् ।\\
विजित्य भार्गवं राममयोध्यां प्राप राघवः ॥ ३९॥
\end{moola}

\begin{moola}
ततः सर्वजनानन्दं कुर्वाणश्चेष्टितैः स्वकैः ।\\
तामध्युवास काकुत्स्थः सीतया सहितः सुखम् ॥ ४०॥
\end{moola}

\end{document}
