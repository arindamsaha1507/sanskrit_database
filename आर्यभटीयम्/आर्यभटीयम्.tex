\documentclass[12pt,a4paper]{report}
\usepackage[utf8]{inputenc}
\usepackage{amsmath}
\usepackage{amsfonts}
\usepackage{amssymb}
\usepackage{polyglossia}

\usepackage[hyphens]{url}
\usepackage{titlesec}
\usepackage[affil-it]{authblk}
\usepackage{titling}

\setmainlanguage{hindi}
\newfontfamily\devanagarifont[Script=Devanagari]{Sanskrit 2003}
% \newfontfamily\devanagarifont[Script=Devanagari]{Lohit Devanagari}
\titleformat*{\section}{\Large\devanagarifont}
\titleformat*{\subsection}{\large\devanagarifont}

\renewcommand{\maketitlehooka}{\devanagarifont}
\renewcommand\Authfont{\fontsize{12}{14.4}\devanagarifont}
\renewcommand\Affilfont{\fontsize{9}{10.8}\itshape}

\newcommand{\devanagarinumeral}[1]{%
  \devanagaridigits{\number\csname c@#1\endcsname}}

% renew all representation of counters
\renewcommand{\thesection}{\devanagarinumeral{section}}
\renewcommand{\thechapter}{\devanagarinumeral{chapter}}
\renewcommand{\thepage}{\devanagarinumeral{page}}
\renewcommand{\theenumi}{\devanagarinumeral{enumi}}

\newenvironment{moola}
{
~\\
\hrule
\begin{center}
\begin{LARGE}
}
{
\end{LARGE}
\end{center}
\hrule
~\\
}


\author{अरिन्दम साहा}
\title{आर्यभटीयम् (भतप्रदीपिका सहित)}
\begin{document}

\maketitle

\chapter{प्रस्तावना}

\section{मङ्गलाचरणम्}

\begin{center}
 \begin{large}
यत्तेजः प्रेरयेत् प्रज्ञां सर्वस्य शशिभूषणं~।\\
मृगटङ्काभवेष्टाङ्कत्रिनेत्रन्तमुपास्महे~॥\\
लीलावती भास्करीयं लघु चान्यच्च मानसं~।\\
व्याख्यातं शिष्यबोधार्थं येन प्राक्तेन चाधुना ~॥\\
तन्त्रस्यार्यभटीयस्य व्याख्याल्पा क्रियते मया~।\\
परमादीश्वराख्येन नाम्नात्र दीपिका~॥\\  
 \end{large}
\end{center}


तत्रायमाचार्य आर्यभटो विघ्नोपशमनार्थं स्वेष्टदेवतानमस्कारं प्रतिपाद्य वस्तुकथनञ्चार्यरूपया करोति~।

\begin{moola}
प्रणिपत्यैकमनेकं कं सत्यां देवतां परं ब्रह्म~।\\
आर्यभटस्त्रीणि गदति गणितं कालक्रियां गोलं~॥१॥\\
\end{moola}

% \begin{quotation}
% ब्रह्मणः अपरम् एकं नाम \underline{क} इति~। तद् \underline{एकमनेकं} तथा च \underline{सत्या} \underline{देवता}~। तत् \underline{परं} \underline{ब्रह्म} \underline{प्रणिपत्य} \underline{आर्यभटः} \underline{त्रीणि} शास्त्राणि \underline{गदति}~। तानि \underline{गणितं} \underline{कालक्रिया} \underline{गोलं} च~।
% \end{quotation}


इति~॥ कं ब्रह्माणं एकं कारणरूपेणैकम् अनेकं कार्यरूपेणानेकं सत्यां देवतां देव एव देवता~। स्वयम्भूरेव पारमार्थिको देव अन्ये तेन सृष्टा इत्यपारमार्थिकाः~। परं ब्रह्म जगतो मूलकारणं त्रिमूर्त्यतीतं सर्वव्याप्तं ब्रह्म स्वयम्भूरित्युक्तो भवति~। आर्यभट एवं ब्रह्माणं प्रणिपत्य गणितं कालक्रियां गोलम् इत्येतानि त्रीणि वस्तूनि निगदति~। परोक्षत्वेन निर्देशान्निगदतीति वचनम्~। तत्र गणितन्नाम सङ्कलितमिश्रश्रेडीदर्शधीकुट्टाकारच्छायाक्षेत्राद्यनेकविधम्~। इह तु कालक्रियागोलयोर्यावन्मात्रं परिकरभूतं तावन्मात्रं सामान्यगणितमेव प्रायशः प्रतिज्ञातं~। अन्यच्च किञ्चित्~। कालस्य क्रिया कालक्रिया~। कालपरिच्छेदोषायभूतं ग्रहगणितं कालक्रियेत्यर्थः~।गोलन्नाम ब्रह्माण्डकटाहमध्यवर्त्याकाशमध्यस्थं ग्रहनक्षत्रकक्ष्यात्मकं स्वमध्यस्यघनवृत्तभूमिकमपक्रमाद्यशेषविशेषोपेतं प्रवहाख्यवायुप्रेरितं कालचक्रज्योतिश्चक्रभपञ्जरादिशब्दवाच्यं गोलः~। स च वृत्तक्षेत्रवाच्चतुरश्राद्यनेकक्षेत्रकल्पनाधारत्वाच्च गणितविशेषगोचर एव~। एतत्त्रयमपि
द्विविधम्~। उपदेशमात्रावसेयन्तन्मूलन्यायावसेयञ्चेति~। तत्र युगप्रमाणमन्दोच्चादिवृत्ताद्यपक्रमाद्युपदेशमात्रावसेयम्~। इष्टदिनग्रहगतीष्टापक्रमस्वाहोरात्रचरदलादिच्छायानाडिकाद्युपदेशसिद्धयुगप्रमाणादितो न्यायावसेयम्~। एवं द्वैविध्यम्~॥ अत्र स्वयम्भूप्रणामकरणेन करिष्यमाणस्य तन्त्रस्य ब्रह्मसिद्धान्तं मूलमितिच प्रदर्शितं~॥

\section{सङ्ख्यापरिभाषा}

अथोपदेशावगम्यान्युगभगणादीन् सङ्क्षेपेण प्रदर्शयितुं दशगीतिकासूत्रं करिष्यन् तदुपयोगिनीं परिभाषामाह~।

\begin{moola}
वर्गाक्षराणि वर्गेऽवर्गेऽवर्गीक्षराणि कात् ङ्मौ यः~।\\
खद्विनवके स्वरा नव वर्गेऽवर्गे नवान्त्यवर्गे वा~॥२॥
 \end{moola}

इति~॥ वर्गाक्षराणि वर्गे~। ककारादीनि मकारातानि वर्गाक्षराणि~। तानि वर्गस्थाने एकशतायुताद्योजस्थाने स्थाप्यानि एवं क्रमेण संख्या वेद्या~॥ अवर्गे
अवर्गाक्षराणि | यकारादीनि अवर्गाक्षराणि | तान्यवर्गस्थाने दशसहस्रलक्षादियुग्मस्थाने स्थाप्यानि~। कात् ककारादारभ्य संख्या वेद्या~। ककार एकसंख्यः
खकारो द्विसंख्य एवं क्रमेण संख्या वेद्या~। ञकारो दशसंख्यः~। टकार एकादशसंख्यः~। नकारो विंशतिसंख्यः | मकारः पञ्चविंशतिसंख्यः~। एवं लिपिपाठक्रमेण संख्या वेद्या~॥ ङ्मौ यः~। ङकारमकारयोर्योगेन तुल्यो यकारः~। पञ्चसंख्यायाः पञ्चविंशतिसंख्यायाश्च योगस्त्रिंशन्संख्य इत्यर्थः~। अत्र प्रथमस्थानमङ्गीकृत्य त्रिंशदित्युक्तं नतु द्वितीयस्थानमकृत्य द्वितीयस्थाने हि त्रिसंख्यो यकारः~। इत्युक्तं भवति~। रेफादयः क्रमेण द्वितीयस्थाने चतुरादिसंख्यास्युः~। हकारो द्वितीयस्थाने दशसंख्यः शतसंख्यावाचक इत्यर्थः~। एवमवर्गस्थानविहितापि हकारसंख्या संख्यान्तरत्वेन वर्गस्थाने स्थाप्यते~। एवं ञकारादिसंख्या वर्गस्थानविहिताप्यवर्गस्थाने संख्यान्तरत्वेन स्थाप्यते~। एतद्धि न्यायतस्सिद्धम्~। अत्र गतुल्यो यकार इति वक्तव्ये ङ्मौ य इति वर्णद्वयेन यदुक्तं तेन संयुक्तैरप्यक्षरैस्संख्या प्रतिपादयिष्यत इति प्रदर्शितं भवति~॥ शून्यभूतानामनङ्गीकृतसंख्याविशेषाणां के प्रयुज्यन्ते~। इत्यत्राह~। खद्विनवके स्वरा नव वर्गे ऽवर्गे~। इति~। द्विनवके अटादशके नव स्वराः क्रमेण प्रयुज्यन्ते~। अ इ उ ऋ ऌ ए ऐ ओ औ~। इत्येते नव स्वराः~। एतदुक्तं भवति~। ककाराद्यक्षरगतास्स्वरास्स्थानप्रदर्शका भवन्ति न संख्याविशेषप्रदर्शका इति~। कथं नवसंख्या अष्टादशके प्रयुज्यन्ते~। इत्यत्राह~। वर्गे ऽवर्गे~। इति~। वर्गस्थानेषु नवस्वकाराया नव स्वराः क्रमेण प्रयुज्यन्ते~। तथा अवर्गस्थानेषु च त एव~। एवमन्यैरपि कल्प्यं~। तथा प्रथमस्वरयुतैर्यकारादिभिर्विहिता संख्या प्रथमे अवर्गस्थाने स्थाप्या~। द्वितीयस्वरयुतैर्द्वितीये अवर्गस्थाने~। एवमन्यैरपीति~। एवमष्टादशस्थानेषु संख्या वेद्या~॥ यदा पुनस्ततो ऽधिकापि संख्या केनचिद्विवक्षिता तदा कथमित्यत्राह | नवान्त्यवर्गे वा~। इति~। नवानां वर्गस्थानानामन्त्ये ऊर्ध्वगते वर्गस्थाननवके तथा नवानामवर्गस्थानानामन्त्ये ऊर्ध्वगते अवर्गस्थाननवके च एते नव स्वरा प्रयुज्यन्ते वा~। केनचिदनुस्वारादिविशेषेण संयुक्ताः प्रयोज्या इत्यर्थः~। शास्त्रव्यवहारस्त्वष्टादशस्थानानि नातिवर्तते~॥

\chapter{गीतिकापादः}

\section{कालगणना}

अथ चतुर्युगे रव्यादीनां भगणसंख्यामाह~।

\begin{moola}
युगरविभगणाः ख्युघृ शशि \\
चयगियिङुशुछ्लृ कु ङिशिबुणॢख्षृ प्राक्~।\\
शनि ढुङ्विघ्व गुरु ख्रि- \\
च्युभ कुज भद्लिझ्नुखृ भृगुबुध सौराः~॥१॥
 \end{moola}

अष्टादशस्थानगतानां संख्यानां संज्ञा तु~\footnote{लीलावत्याम्}

\begin{center}
 \begin{large}
एकदशशतसहस्रायुतलक्षप्रयुतकोटयः क्रमशः~।\\
अर्बुदमब्जं खर्वनिखर्वमहापद्मशङ्कवस्तस्मात्~॥\\
जलधिञ्चान्त्यं मध्यं परार्धमिति दशगुणोत्तरं संज्ञाः~।
 \end{large}
\end{center}

इत्यनेन वेद्या~॥ युगरविभगणाः~। चतुर्युगे रवेर्भगणाः ख्युघृ इति~। उकारयुतखकारेणायुतद्वयमुक्तं~। उकारयुतयकारेण लक्षत्रयं~। एवं सर्वत्र हल्द्वये
एक एव स्वर उभयत्र सम्बध्यते~। ऋकारयुतघकारेण प्रयुतचतुष्कं~। एवमनेन न्यायेन सर्वत्र संख्या वेद्या~। शशि~। शशिन इत्यर्थः~। सूत्रे  यद्विभक्तिकोऽपि प्रयोगस्स्यात्~। चयगियिङुशुछ्लृ इति युगभगणाश्शशिनः~। च षठ्~। य त्रिंशत्~। गि त्रिशतं~। यि त्रिसहस्रं~। ङु युतपञ्चकं~। शु लक्षसप्तकं~। छृ प्रयुतसप्तकं~। लृ कोटिपञ्चकं~। इति~॥ कु~। भूमेरित्यर्थः~। ङिशिवुणॢख्षृ इति भगणाः~। प्राक् प्राग्गत्या सम्भूता भगणा इत्यर्थः~। णॢ पञ्चदशार्बुदं | नवमस्थाने पञ्च दशमस्थाने एकञ्चेत्यर्थः~। खृ प्रयुतद्वयं~। षृ कोट्यष्टकं~। भूमेर्यत्प्राङ्मुखं भ्रमणं तस्य चतुर्युगे संभूता संख्यात्रोक्ता~। भूमिर्यदचलेति प्रसिद्धा तस्याः कथमत्र भ्रमणकथनं~। उच्यते~। प्रवहाक्षेपात्पश्चिमाभिमुखं भ्रमतो नक्षत्रमण्डलस्य मिथ्याज्ञानवशाद्भूमेर्भ्रमणं प्रतीते~। तदङ्गीकृत्येह भूमेर्भ्रमणमुक्तं~। वस्तुतस्तु न भूमेर्भ्रमणमस्ति~। अतो नक्षत्रमण्डलस्य भ्रमणप्रदर्शनपरमत्र भूभ्रमणकथनमिति वेद्यं~। वक्ष्यतिच मिथ्याज्ञानं~\footnote{गोलपादे}

\begin{center}
 \begin{large}
अनुलोमगतिर्नौस्थः पश्यत्यचलं विलोमगं यद्वत्~।\\
अचलानि भानि समपश्चिमगानि लङ्कायां~॥
 \end{large}
\end{center}

इति~। अहोरात्रेण हि भगोलस्य समस्तभागभ्रमणादूर्ध्वं रवेर्दिनगतितुल्यभागोऽपि भ्रमति~। अतो रवेर्युगभगणयुतभूदिवसैस्तुल्या नक्षत्रमण्डलस्य भ्रमणमितिर्भवति~। सैवात्रोक्ता स्यात्~॥ शनि ढुङ्विघ्व इति~। शनेर्युगभगणाः~। ढु अयुतानाञ्चतुर्दश | ङि पञ्चशतं~। वि षट्सहस्रं | घ चत्वारि | व षष्टिः~॥ गुरु ख्रिच्युभ इति~। गुरोर्भगणाः~। खि इति द्विशतं~। रि इति चतुस्सहस्रं~। चु इत्ययुतषट्कं~। यु इति लक्षत्रयं~। भ इति चतुर्विंशतिः~॥ कुज भद्लिझ्नुखृ इति~। कुजस्य भगणाः~। भ चतुर्विंशतिः~। दि अष्टशताधिकसहस्रं~। लि पञ्चसहस्रं | झु अयुतनवकं~। नु लक्षद्वयं~। खृ प्रयुतदयं~। अत्र संख्यायोगे भगणसिद्धिः~॥ भृगुबुध सौराः~। भृगुबुधयोर्युगभगणास्सौरा एव~। सूर्यगभगणाः ख्युघृ एव~॥

एवं प्रथमसूत्रेण रव्यादीनां युगभगणान् प्रदर्श्य द्वितीयसूत्रेण चन्द्रोच्चभगणान् बुधभृग्वोश्शीघ्रोच्चभगणांश्च शेषाणां कुजगुरुशनैश्चराणां शीघ्रोच्चञ्च चन्द्रपातभगणांच भगणारम्भकालञ्चाह~।

\begin{moola}
चन्द्रोच्च ज्रुष्खिध बुध \\
सुगुशिथृन भृगु जषबिखुछृ शेषार्काः~। \\
बुफिनच पातविलोमा \\
बुधाह्न्यजार्कोदयाच्च लङ्कायां~॥ २~॥
\end{moola}

चन्द्रोच्चस्य ज्रुष्खिध इति भगणाः~। र्जुष्खिध इति वा पाठः~। जु अयुताष्टकं~। रु लक्षचतुष्कं~। षि अष्टसहस्रं~। खि द्विशतं~। ध एकोनविंशतिः~॥  बुधस्य
शीघ्रोच्चभगणाः सुगुशिथृन इति~। सु लक्षनवकं~। गुं अयुतत्रयं~। शि सप्तसहस्रं~। थृ प्रयुतसप्तदशकं~। न विंशतिः~॥ भृगोश्शीघ्रोच्चभगणा जषबिखुछृ इति~। ज अष्टौ~। ष अशीतिः~। बि शतत्रयाधिकद्विसहस्रं | खु अयुतदयं~। छृ प्रयुतसप्तकं~॥ शेषार्काः~। शेषाणां कुजगुरुमन्दानां शीघ्रोच्चभगणा आर्काः~। अर्कभगणा एव~। उपरिष्टादेषां मन्दोच्चांशान्वक्ष्यति~। अत इहोक्ताश्शीघ्रोच्चभगणा इति सिध्यति~। बुफिनच इति पातस्य चन्द्रपातस्य विलोमात्मकभगणाः~। बु अयुतानां त्रयोविंशतिः~। फि शतद्वयाधिकसहस्रद्वयं~। न विंशतिः~। च षट्~॥ कुजादीनां पातभगणान्वक्ष्यति~। अर्कस्य तु विक्षेपो न विधीयते~। अत एते चन्द्रपातस्य भगणा इति सिध्यति~। उच्चपातानां व्योम्नि दर्शनं नास्ति~। तथाच ब्रह्मगुप्तः

\begin{center}
 \begin{large}
प्रतिपादनार्थमुच्चाः प्रकल्पिता ग्रहगतेस्तथा पाताः~।
\end{large}
\end{center}

इति~॥ बुधाह्न्यजार्कोदयाच्च लङ्कायां~। कृतयुगादौ बुधवारे लङ्कायां सूर्योदयमारभ्य~। अजात् मेषादिमारभ्य राशिचक्रे गच्छतां रव्यादीनां भगणा अत्रोक्ता इत्यर्थः~। सूर्योदयो मध्यसूर्योदयः कल्पारम्भस्तु स्फुटसूर्योदयः~। तत्र मध्यमस्फुटयोर्विशेषाभावात्~॥ 

कल्पकालान्तर्गतमनून् गतकालञ्च तृतीयसूत्रेणाह~।

 \begin{moola}
काहोमनवो ढ मनुयुग \\
श्ख गतास्ते च मनुयुग छ्ना च~। \\
कल्पादेर्युगपादा \\
ग च गुरुदिवसाच्च भारतात्पूर्वम्~॥३॥
\end{moola}

काहोमनवो ढ~। क कस्य ब्रह्मणः~। अहः अद्भि मनवो ढ चतुर्दश भवन्ति~। मनुयुग श्ख~। एकैकस्य मनोः काले युगानि चतुर्युगाणि श्ख | श सप्ततिः~। ख द्वयं~। द्वासप्ततिरित्यर्थः~। गतास्ते च~। एतस्माद्वर्तमानात्कलियुगात्पूर्वमतीतास्ते मनवः~। च षट्~। मनुयुग छ्ना च~। वर्तमानस्य सप्तमस्य मनोः~। अतीतानि चतुर्युगाणि छ्ना~। छा सप्त~। ना विंशतिः~। सप्तविंशतिरित्यर्थः~। स्वराणां ह्रस्वदीर्घयोर्न विशेषः~। अकारसदृश एवाकार~॥ कल्पादेर्युगपादा ग च गुरुदिवसाच्च भारतात्पूर्वं~। युगपादा ग च~। वर्तमानस्याष्टाविंशस्य चतुर्युगस्य ग पादाश्च~। त्रयः पादाश्च~। गता भवन्ति~। अस्मिन्सूत्रे ऽनाद्यं चकारत्रयं न संख्याप्रदर्शकं~॥ कदा एवमित्यत्राह~। कल्पादेर्भारताद्गुरुदिवसात्पूर्वमिति~। भारता युधिष्ठिरादयः~। तैरुपलक्षितो गुरुदिवसो भारतगुरुदिवसः~। राज्यं चरतां युधिष्ठिरादीनामन्त्यो गुरुदिवसो द्वापरावसानगत इत्यर्थः~। तस्मिन्दिने युधिष्ठिरादयो राज्यमुत्सृज्य महाप्रस्थानं गता इति प्रसिद्धिः~। तस्मारुदिवसात्पूर्वं कल्पादेरारभ्य गता मन्वादय इहोक्ताः~। इत्यर्थः~। अस्मिन्पक्षे युगानि परस्परसमानि युगपादश्च चतुर्युगचतुर्थांशः~। अन्यथा चेत् बुधवारादिके चतुर्युगे कलियुगारम्भश्शुक्रवारे न संभवति। अतः कृतयुगारम्भो बुधवार इति~। बुधाह्न्यजार्केदयाच्च लङ्कायामिति~। पठिताश्च प्रकाशिकायां कलियुगादेः प्रागतीताः कल्पदिवसाः शराश्विषट्खाद्रिशराद्रिवेदकृतेषुयुग्मखरसंमितः स्यात्~। इति~। अहर्गणो नात्र विशेष्यः~। अनेनापि युगानां समयस्सिध्यति~॥ 

\section{दिग्गणना}

चतुर्थेन सूत्रेण राश्यादिविभागमाकाशकक्ष्यायोजनप्रमाणं प्राणकलयोः क्षेत्रसाम्यं ग्रहनक्षत्रकक्ष्यायोजनप्रमाणञ्चाह~।

\begin{moola}
शशिराशयष्ठ चक्रं \\
तेऽंशकलायोजनानि यवञगुणाः~।\\
प्राणेनैति कलां भं \\ 
खयुगांशे ग्रहजवो भवांशेऽर्कः~॥ ४~॥
\end{moola}

शशिनश्चक्रं भगणा द्वादशगुणिता राशयः~। शशिनो युगभगणा द्वादशगुणिता युगराशयो भवन्ति~। भगणाद् द्वादशांशो राशिरित्युक्तं भवति~। ते राशयो यगुभगणास्त्रिंशगुणिता अंशा भवन्ति~। राशेस्त्रिंशांशो भाग इत्युक्तं भवति~। ते ऽंशा वगुणाष्षष्टिगुणाः कला भवन्ति~। अंशात् षष्ठ्यंशः कलेत्युक्तं भवति~। ताः कला ञगुणा योजनानि भवन्ति~। शशिनो युगभवाः कला दशगुणिता आकाशकक्ष्यायोजनानि भवन्तीत्यर्थः~। ब्रह्माण्डकटाहावच्छिन्नस्य सूर्यरश्मिव्याप्तस्याकाशमण्डलस्य परिधियोजनान्याकाशकक्ष्यायोजनानीत्युच्यन्ते~। खखषष्द्यद्रीषुखाश्विस्वराध्यद्य्रब्धिभास्करा इत्याकाशकक्ष्यायोजनानि~॥ प्राणेनैति कलां भं~। प्राणेनोच्छ्वासतुल्येन कालेन भं ज्योतिश्चक्रं कलामेति कलापरिमितं प्रदेशं प्रवहवायुवशात्पश्चिमाभिमुखं गच्छति~। खखषड्भूयमतुल्या हि ज्योतिश्चक्रगताः कलाः~। चक्रभ्रमणकालनिष्पन्नाः प्राणाश्च तत्तुल्या इत्युक्तं भवति~। अतो घटिकामण्डलगताः प्राणा राशिचक्रगताः कलाश्च क्षेत्रतस्तुल्या इति चोक्तं भवति~॥ खयुगांशे ग्रहजवः~। खमाकाशकक्ष्या~। युगं ग्रहस्य भगणाः~। आकाशकक्ष्यातो ग्रहभगणैराप्तं ग्रहजवः~। एकपरिवृत्तौ ग्रहस्य जवो गतिमानं योजनात्मकं भवति। ग्रहस्य कक्ष्यामण्डलपरिधियोजनमित्यर्थः~॥ भवांशे ऽर्कः~। भस्य नक्षत्रमण्डलस्य कक्ष्याया वांशे षष्ठ्यंशे अर्को भ्रमति~। नक्षत्रकक्ष्यातष्षष्ठ्यंशेन तुलितार्ककक्ष्येत्युक्तं भवति~। अत्र नक्षत्रकक्ष्या विधीयते~। अर्ककक्ष्या हि पूर्वविधिनैव सिद्धा~। अर्ककक्ष्या षष्टिगुणिता नक्षत्रकक्ष्या भवतीत्युक्तं भवति~॥

पञ्चमेन योजनपरिमितिं भूम्यादेजनप्रमाणञ्च प्रदर्शयति~।

\begin{moola}
नृषि योजनं ञिला भू \\
व्यासोऽर्केन्द्वोर्घ्रिञा गिण क मेरोः~। \\
भृगुगुरुबुधशनिभीमा: \\
शशि ङञणनमांशकास्समार्कसमाः~॥५॥
\end{moola}

नृषि योजनं~। नृ नरप्रमाणानां षि अष्टसहस्रं योजनं योजनस्य प्रमाणं भवति~॥ ञिला भूव्यासः~। ञि सहस्रं ला पञ्चाशत्~। एतानि भूमेर्व्यासप्रमाणयोजनानि॥ श्रर्केञिा गिण~। अर्कमण्डलस्य व्यासप्रमाणयोजनानि घिञा इति~। घि चवारि शतानि~। रि चवारि सहस्राणि~। ञ दश~। इन्दोर्गिण इति~। गि त्रिशतं~। ण पञ्चदश~॥ क मेरोः~। मेरोर्व्यासयोजनप्रमाणं क। एकमित्यर्थः~॥ भृग्वादीनां बिम्बयोजनानि क्रमाच्छशिनो बिम्बस्य योजनव्यासात् ङांशञांशणांशनांशमांशतुल्यानि~। पञ्चांशदशांशपञ्चदशांशविंशांशपञ्चविंशांशतुल्यानीत्यर्थः~॥ शशिकक्ष्यासाधिता ते व्यासाः~। अतो विष्कम्भार्धहताश्चन्द्रस्य योजनकर्णभक्ता लिप्ता भवन्ति~। पुनरपि ता विष्कम्भार्धहतास्स्वस्वमन्दकर्णशीघ्रकर्णयो यौगार्धकृतास्स्फुटा भवन्ति~। इत्युपदेशः~। तथाच\footnote{सूर्यसिद्धान्ते ग्रहयुत्यधिकारे १४}

\begin{center}
 \begin{large}
त्रिचतुः कर्णयुत्याप्तास्ते द्विघ्नास्त्रिज्यया हताः~।
\end{large}
\end{center}

इति~। अत्र चन्द्रस्य योजनकर्णश्चन्द्रस्य मध्ययोजनकर्णः~॥ समार्कसमाः~। युगसमा युगार्कभगणसमा इत्यर्थः~॥ 

ग्रहाणां विषुवत उत्तरेण दक्षिणेन चापयानप्रमाणं पुरुषप्रमाणञ्च षष्ठेन सूत्रेणाह~।

\begin{moola}
 भाऽपक्रमो ग्रहांशाः \\
 शशिविक्षेपोऽपमण्डलाज्झार्धम्~। \\
शनिगुरुकुज खकगार्धं \\
भृगुबुध ख स्चाङ्गुलो घहस्तो ना~॥६॥
\end{moola}

भाऽपक्रमो ग्रहांशाः। ग्रहाणां भ अंशाश्चतुर्विंशतिभागा अपक्रमः~। परमापक्रम इत्यर्थः~। पूर्वापरस्वस्तिकात्रिराश्यन्तरे घटिकामण्डलापक्रममण्डलयोरन्तरालं
चतुर्विंशतिभागतुल्यमित्यर्घः~॥ अपमण्डलाच्छशिनः परमविक्षेपो झार्धं नवानामर्ध सार्धाश्चत्वारोऽंशाः~॥ शनिगुरुकुज खकगार्थ~। शनेर्विक्षेपः ख द्वावंशौ~।
गुरोः क एकांशः~। कुजस्य गार्ध त्रयाणामर्धं सार्धोऽंशः~। भृगुबुध ख~। भृगुबुधयोर्विक्षेपः ख द्वावंशौ~॥ स्चाङ्गुल घस्तो ना~। पुरुषस्स्चाङ्गुलो घहस्तश्च~।
स नवतिः। च षट्। षण्णवत्यङ्गुलः पुरुषः~। घहस्तश्चतुर्हस्तश्च पुरुषः। नृषियोजनमित्यादौ नरशब्देन षण्णवत्यङ्गुलप्रमाणमुदितमित्युक्तं भवति~। तदेव
चतुर्हस्तप्रमाणं भवति। चतुर्विंशत्यङ्गुलैरेको हस्तो भवतीति चोक्तं भवति~। अङ्गुलस्य परिमाणानुपदेशाल्लोकसिद्धमेवाङ्गुलं गृह्यते~। उक्तञ्च तत्परिमाणं
तन्त्रान्तरे (लीलावत्यां)

\begin{center}
 \begin{large}
यवोदरैरङ्गुलमष्टसंख्यैर्हस्तोऽङ्गुलैष्षङ्गुणितैश्चतुर्भिः~। \\
हस्तैश्चतुर्भिर्भवतीह दण्डः क्रोशस्सहस्रद्वितयेन तेषां~॥
\end{large}
\end{center}

इति~॥ इह विक्षेपकथने शन्यादीनां भृगुबुधवाश्च पृथग्ग्रहणं कृतं~। तेन तेषां तयोश्च विक्षेपानयने प्रकारभेदोऽस्तीति सूचितं~॥ 

कुजादीनां पञ्चानां पातभागान् सूर्ययुतानां तेषां मन्दोच्चांशांश्च सप्तमेन सूत्रेणाह~।

\begin{moola}
बुधभृगुकुजगुरुशनि नव \\
रषहा गत्वांशकान्प्रथमपाताः । \\
सवितुरमीषाञ्च तथा \\
द्वा ञखि सा ह्दा ह्लय खिच्य मन्दोच्चं ॥७॥ 
\end{moola}

बुधस्य पातांशाः न विंशतिः । भृगोः व षष्टिः । कुजस्य र चत्वारिंशत् । गुरोः ष अशीतिः । शनेः ह शतं । गवांशकान्प्रथमपाताः । उक्तानेतानेवांशकान्मेषादितो गत्वा व्यवस्थिता बुधादीनां प्रथमपातास्स्युः । प्रथमशब्देन द्वितीयोऽपि पातो ऽस्तीति सूचितं । स च प्रथमपाताच्चक्राधान्तरे स्थितस्स्यात् । विक्षेपमण्डलापमण्डलयोस्संपातस्थानं पातशब्देनोच्यते । तदुभयत्र भवति । गत्वेतिवचनात्तेषां पातानां गतिरभिप्रेता । गतिश्च विलोमा । पातविलोमा इत्यनेन पातानां विलोमगत्वमुक्तं । अस्मिन्काले पातानां स्थितिरेवमित्युक्तं भवति ॥ सवितुर्मन्दोच्चं तथा द्वा। दा अष्टादश । वा षष्टिः । अष्टसप्ततिभागान् तथा मेषादितो गत्वा स्थितं सवितुर्मन्दोच्चमित्यर्थः । अमीषामुक्तानां बुधादीनां मन्दोच्चानि ञखिरित्येवमादिभिरुक्तानि । बुधस्य मन्दोच्चं ञखि दशाधिकशत- द्वयभागाः । भृगोः सा नवतिभागाः । कुजस्य ह्दा । हा शतं दा अष्टादश । अष्टादशाधिकशतभागाः। गुरोः ह्लय । ह शतं ल पञ्चाशत् व त्रिंशत् । अशीत्यधिकशतभागाः। शनेः खिच्य । खि शतद्वयं च षट् य त्रिंशत् । षट्त्रिंशदुत्तरशतद्वयभागाः। गवेतिवचनादेषामपि गतिरभिहिता । गतिश्चानुलोमा चन्द्रोच्चवत् । अस्मिन्काल एव मन्दाञ्चस्थितिरित्युक्तं भवति । पातोच्चानां बहुना


\end{document}
